\documentclass[]{article}
\usepackage{lmodern}
\usepackage{amssymb,amsmath}
\usepackage{ifxetex,ifluatex}
\usepackage{fixltx2e} % provides \textsubscript
\ifnum 0\ifxetex 1\fi\ifluatex 1\fi=0 % if pdftex
  \usepackage[T1]{fontenc}
  \usepackage[utf8]{inputenc}
\else % if luatex or xelatex
  \ifxetex
    \usepackage{mathspec}
  \else
    \usepackage{fontspec}
  \fi
  \defaultfontfeatures{Ligatures=TeX,Scale=MatchLowercase}
\fi
% use upquote if available, for straight quotes in verbatim environments
\IfFileExists{upquote.sty}{\usepackage{upquote}}{}
% use microtype if available
\IfFileExists{microtype.sty}{%
\usepackage{microtype}
\UseMicrotypeSet[protrusion]{basicmath} % disable protrusion for tt fonts
}{}
\usepackage[margin=1in]{geometry}
\usepackage{hyperref}
\hypersetup{unicode=true,
            pdftitle={Tropical wet secondary forest regeneration patterns},
            pdfauthor={Nicholas L Medina},
            pdfborder={0 0 0},
            breaklinks=true}
\urlstyle{same}  % don't use monospace font for urls
\usepackage{graphicx,grffile}
\makeatletter
\def\maxwidth{\ifdim\Gin@nat@width>\linewidth\linewidth\else\Gin@nat@width\fi}
\def\maxheight{\ifdim\Gin@nat@height>\textheight\textheight\else\Gin@nat@height\fi}
\makeatother
% Scale images if necessary, so that they will not overflow the page
% margins by default, and it is still possible to overwrite the defaults
% using explicit options in \includegraphics[width, height, ...]{}
\setkeys{Gin}{width=\maxwidth,height=\maxheight,keepaspectratio}
\IfFileExists{parskip.sty}{%
\usepackage{parskip}
}{% else
\setlength{\parindent}{0pt}
\setlength{\parskip}{6pt plus 2pt minus 1pt}
}
\setlength{\emergencystretch}{3em}  % prevent overfull lines
\providecommand{\tightlist}{%
  \setlength{\itemsep}{0pt}\setlength{\parskip}{0pt}}
\setcounter{secnumdepth}{0}
% Redefines (sub)paragraphs to behave more like sections
\ifx\paragraph\undefined\else
\let\oldparagraph\paragraph
\renewcommand{\paragraph}[1]{\oldparagraph{#1}\mbox{}}
\fi
\ifx\subparagraph\undefined\else
\let\oldsubparagraph\subparagraph
\renewcommand{\subparagraph}[1]{\oldsubparagraph{#1}\mbox{}}
\fi

%%% Use protect on footnotes to avoid problems with footnotes in titles
\let\rmarkdownfootnote\footnote%
\def\footnote{\protect\rmarkdownfootnote}

%%% Change title format to be more compact
\usepackage{titling}

% Create subtitle command for use in maketitle
\providecommand{\subtitle}[1]{
  \posttitle{
    \begin{center}\large#1\end{center}
    }
}

\setlength{\droptitle}{-2em}

  \title{Tropical wet secondary forest regeneration patterns}
    \pretitle{\vspace{\droptitle}\centering\huge}
  \posttitle{\par}
    \author{Nicholas L Medina}
    \preauthor{\centering\large\emph}
  \postauthor{\par}
      \predate{\centering\large\emph}
  \postdate{\par}
    \date{29/12/2018}


\begin{document}
\maketitle

\hypertarget{introduction}{%
\section{Introduction}\label{introduction}}

Primary tropical forests are idolized for their - role/importance of
secondary forests

\hypertarget{community-composition}{%
\subsection{\texorpdfstring{\emph{Community
composition}}{Community composition}}\label{community-composition}}

\begin{itemize}
\item
  factors affecting their aboveground biomass accumulation Aboveground
  biomass accumulation is likely affected by interspecific interactions,
  namely among N-fixers and non-N-fixers. While N-fixers are thought to
  facilitate the growth of other plants (CITATIONS FROM YOUNG AG),
  recent data suggest that competition, such as for P, may be more
  important for interspecific interactions than expected facilitation by
  N supply (Taylor et al 2017). This makes sense, given that P
  availability, rather than that of N, is more limiting in tropical
  ecosystems compared to temperate systems, especially given that they
  are estimated to have sufficient N to support decadal aboveground
  biomass accumulation (Brookshire et al 2019). Furthermore, with regard
  to nutrient dynamics, N-fixing trees in tropical forests are also
  expected to be major sources of N2O (Kou-Giesbrecht \& Menge 2019),
  further offsetting biomass balances.
\item
  \begin{itemize}
  \tightlist
  \item
    limiting nutrients likely fluctuating/oscillatory (Sullivan et al
    2014)
  \end{itemize}
\item
  \begin{itemize}
  \tightlist
  \item
    interactions and effects sizes are likely variable at species-level
    (Keller et al 2013; Gei \& Powers 2013)
  \end{itemize}
\item
  factors affecting their species coexistence
\item
  \begin{itemize}
  \tightlist
  \item
    relevance of stage structuring by specific gravity? (Nature EE 2019
    2ndFOR)
  \end{itemize}
\item
  \begin{itemize}
  \tightlist
  \item
    predicted distributions of species/wood densities
  \end{itemize}
\end{itemize}

\hypertarget{this-study}{%
\subsection{\texorpdfstring{\emph{This
study}}{This study}}\label{this-study}}

\begin{itemize}
\tightlist
\item
  census of \textasciitilde{}20yo secondary forest derived from an
  abandoned {[}common name for Bombacopsis{]} plantation. The base
  species is known to do well in dry forests, but not in the wet lowland
  forests like on the Osa Peninsula.
\end{itemize}

\hypertarget{methods}{%
\section{Methods}\label{methods}}

\hypertarget{study-site}{%
\subsection{\texorpdfstring{\emph{Study
site}}{Study site}}\label{study-site}}

This study was set in the lowland wet tropical rainforest of the Osa
Peninsula in SW Puntarenas, Costa Rica. The site is property of Osa
Conservation (osaconservation.org) located at the Greg Gund Conservation
Center. The censused parcel was 20 ha and undergoing secondary
succession for \textasciitilde{}20 years after abandonment as a wood
plantation using \emph{Bombacopsis quinata}, which unfortunately is
better suited to the dry forests of Guanacaste and the Nicoya Peninsula
(CITATION). Trees ≥10 cm in diameter were\ldots{}The study was done
during the rainy season months June - August 2013.

\hypertarget{census-design}{%
\subsection{\texorpdfstring{\emph{Census
design}}{Census design}}\label{census-design}}

Geographical information systems software (ArcGIS, WEBSITE, and QGIS,
{[}WEBSITE{]}) were used to create a polygon of the 20-ha parcel, which
was divided into six 50-m discrete bands (0-300m) that represented a
gradient of distance from the known adjacent primary forest (Fig S1).
This distance was made discrete as a simpler alternative to finding the
shortest linear distance between the plot and primary forest edge. The
number of plots set up in each band was coarsely scaled to the of area
each distance bance, such that the progression of plots contained in
each band from closest to furthest from primary forest was 10, 8, 5, 3,
and 1 respectively. Each plot was 21 \emph{x} 21m and set up June -
August 2013.

In each plot, every stem ≥10cm in {[}diameter/circumference?{]} was
measured for diameter at breast height (DBH, breast height =
\textasciitilde{}2.5 - 3.1m) and total height. Tree height was measured
by using a rangefinder (COMPANY/BRAND Bushnell?) to measure from an
observer to the crown (where it was visible), and then to the trunk at
breast height; these 2 distances were then used to triangulate and
calculate the tree's height using Pythagoras's Theorem. Following
typical forestry guidelines, in cases where a tree split into ≥2 stems
at breast height, each stem was measured separately; in cases where it
only initially split above breast height, it was measured as 1 stem.

\hypertarget{aboveground-biomass}{%
\subsection{\texorpdfstring{\emph{Aboveground
biomass}}{Aboveground biomass}}\label{aboveground-biomass}}

Aboveground biomass was estimated for each stem using the allometric
equation developed by Chave \emph{et al} (2014), which performed similar
to those in Chave \emph{et al} (2005) developed specifically for wet
tropical forests:

\emph{AGB = 0.0673 x (p D\^{}2 H)\^{}0.976}

with \emph{D} in cm, \emph{H} in m, and \emph{p} in g cm\^{}-3. Measured
DBHs and heights were used, along with wood densities ( \emph{p} )
exracted from the literature. In most cases, these values were only
available at the genus level, and unrepresented taxa were assumed to be
0.58 as stated by the World Agroforestry Database (CITE). References
used for this step are given in \emph{Table S1}.

\hypertarget{abiotic-plot-characteristics}{%
\subsection{\texorpdfstring{\emph{Abiotic plot
characteristics}}{Abiotic plot characteristics}}\label{abiotic-plot-characteristics}}

\hypertarget{canopy-cover}{%
\subsubsection{\texorpdfstring{\emph{Canopy
cover}}{Canopy cover}}\label{canopy-cover}}

Light reaching the forest floor was measured at the center of each plot,
at chest height, using a densiometer (BRAND). According to manufacturer
instructions, the final value used for the plot was an average of 4
different readings taken facing each of the 4 cardinal directions:
North, South, East, and West.

\hypertarget{topography}{%
\subsubsection{\texorpdfstring{\emph{Topography}}{Topography}}\label{topography}}

The slope of the forest floor in each plot was measured using {[}a
rangefinder by measuring the distance from one (specifc?) corner of the
plot to another (specific?) corner.{]}

\hypertarget{statistical-analysis}{%
\subsection{\texorpdfstring{\emph{Statistical
analysis}}{Statistical analysis}}\label{statistical-analysis}}

Analysis ideas from lit: - \ldots{}shade species may do better here
vs.~clear-cutting?

\hypertarget{census-structure}{%
\subsubsection{\texorpdfstring{\emph{Census
structure}}{Census structure}}\label{census-structure}}

\begin{verbatim}
##             Df Sum Sq Mean Sq F value Pr(>F)  
## dist         1    0.1    0.12   0.004 0.9500  
## plot         1  142.5  142.47   4.753 0.0428 *
## dist:plot    1    6.8    6.81   0.227 0.6393  
## Residuals   18  539.6   29.98                 
## ---
## Signif. codes:  0 '***' 0.001 '**' 0.01 '*' 0.05 '.' 0.1 ' ' 1
\end{verbatim}

\emph{Table S\#.} ANOVA results for \emph{B. quinata} plot stem density
by distance from primary forest edge.

\begin{verbatim}
##             Df Sum Sq Mean Sq F value Pr(>F)  
## dist         1   22.6    22.6   0.329  0.571  
## plot         1  331.4   331.4   4.817  0.037 *
## Residuals   27 1857.4    68.8                 
## ---
## Signif. codes:  0 '***' 0.001 '**' 0.01 '*' 0.05 '.' 0.1 ' ' 1
\end{verbatim}

\begin{verbatim}
##             Df Sum Sq Mean Sq F value Pr(>F)
## dist         1    0.0    0.02   0.001  0.978
## plot         1    0.0    0.00   0.000  0.994
## Residuals   27  859.9   31.85
\end{verbatim}

\begin{verbatim}
##             Df Sum Sq Mean Sq F value Pr(>F)
## dist         1    365   365.5   1.280  0.268
## plot         1    439   439.1   1.537  0.226
## Residuals   27   7712   285.6
\end{verbatim}

\emph{Table S\#.} ANOVA results for environmental variation in (A) light
and (B) terrain slope by distance from primary forest edge.

\hypertarget{canopy-structure}{%
\subsubsection{\texorpdfstring{\emph{Canopy
structure}}{Canopy structure}}\label{canopy-structure}}

\begin{itemize}
\tightlist
\item
  height and {[}mean{]} SPG (Díaz et al 2016) by dist, or PCoA
\end{itemize}

\begin{verbatim}
##           Df  Pillai approx F num Df den Df   Pr(>F)   
## dist       1 0.40709   8.5824      2     25 0.001453 **
## plot       1 0.22378   3.6036      2     25 0.042153 * 
## n          1 0.00828   0.1044      2     25 0.901292   
## Residuals 26                                           
## ---
## Signif. codes:  0 '***' 0.001 '**' 0.01 '*' 0.05 '.' 0.1 ' ' 1
\end{verbatim}

\emph{Table S\#}. MANOVA results showing effects of distance from
primary forest edge and plot ID on mean plot height and wood density.

\begin{verbatim}
##              Df Sum Sq Mean Sq F value   Pr(>F)    
## dist          1   85.4   85.42  14.549 0.000171 ***
## plot          1   19.7   19.66   3.349 0.068395 .  
## dist:plot     1   23.8   23.76   4.047 0.045278 *  
## Residuals   261 1532.5    5.87                     
## ---
## Signif. codes:  0 '***' 0.001 '**' 0.01 '*' 0.05 '.' 0.1 ' ' 1
\end{verbatim}

\begin{verbatim}
##             Df Sum Sq Mean Sq F value Pr(>F)  
## dist         1   9.34   9.338   4.375 0.0509 .
## plot         1   0.31   0.307   0.144 0.7090  
## n            1   2.73   2.725   1.277 0.2733  
## Residuals   18  38.42   2.134                 
## ---
## Signif. codes:  0 '***' 0.001 '**' 0.01 '*' 0.05 '.' 0.1 ' ' 1
\end{verbatim}

\emph{Table S\#.} ANOVA results at (A) tree and (B) plot levels for
\emph{B. quinata} height by distance from primary forest edge.

\begin{verbatim}
## 
## Call:
## lm(formula = kg14 ~ n + N + W + luz + incl, data = datp)
## 
## Residuals:
##    Min     1Q Median     3Q    Max 
## -82.00 -32.71  -1.70  16.90 128.52 
## 
## Coefficients:
##               Estimate Std. Error t value Pr(>|t|)  
## (Intercept) -2.934e+05  7.794e+05  -0.376   0.7099  
## n           -1.803e+00  9.241e-01  -1.951   0.0629 .
## N            2.144e+04  1.306e+04   1.643   0.1135  
## W            1.359e+03  9.081e+03   0.150   0.8823  
## luz          2.692e+00  2.467e+00   1.091   0.2861  
## incl         7.389e-01  6.476e-01   1.141   0.2651  
## ---
## Signif. codes:  0 '***' 0.001 '**' 0.01 '*' 0.05 '.' 0.1 ' ' 1
## 
## Residual standard error: 55.98 on 24 degrees of freedom
## Multiple R-squared:  0.2004, Adjusted R-squared:  0.03387 
## F-statistic: 1.203 on 5 and 24 DF,  p-value: 0.3375
\end{verbatim}

\begin{verbatim}
## 
## Call:
## lm(formula = kg14 ~ n + N + W + luz + incl, data = datpBno)
## 
## Residuals:
##    Min     1Q Median     3Q    Max 
## -91.13 -44.18 -11.50  23.32 156.35 
## 
## Coefficients:
##               Estimate Std. Error t value Pr(>|t|)  
## (Intercept) -1.149e+06  9.813e+05  -1.171   0.2532  
## n           -3.177e+00  1.639e+00  -1.939   0.0644 .
## N            2.986e+04  1.591e+04   1.877   0.0727 .
## W            1.078e+04  1.143e+04   0.943   0.3552  
## luz          5.138e+00  3.113e+00   1.651   0.1118  
## incl         1.101e+00  8.582e-01   1.283   0.2119  
## ---
## Signif. codes:  0 '***' 0.001 '**' 0.01 '*' 0.05 '.' 0.1 ' ' 1
## 
## Residual standard error: 70.37 on 24 degrees of freedom
## Multiple R-squared:  0.2459, Adjusted R-squared:  0.08883 
## F-statistic: 1.565 on 5 and 24 DF,  p-value: 0.2077
\end{verbatim}

\begin{verbatim}
## 
## Call:
## lm(formula = kg14 ~ n + N + W + luz + incl, data = datpB)
## 
## Residuals:
##     Min      1Q  Median      3Q     Max 
## -68.473 -22.095  -0.378  18.756 121.415 
## 
## Coefficients:
##               Estimate Std. Error t value Pr(>|t|)   
## (Intercept)  2.490e+06  7.232e+05   3.443  0.00334 **
## n           -2.528e-01  1.867e+00  -0.135  0.89397   
## N           -1.247e+04  1.405e+04  -0.888  0.38788   
## W           -2.863e+04  8.329e+03  -3.437  0.00338 **
## luz         -4.005e+00  2.642e+00  -1.516  0.14906   
## incl         4.903e-01  6.922e-01   0.708  0.48894   
## ---
## Signif. codes:  0 '***' 0.001 '**' 0.01 '*' 0.05 '.' 0.1 ' ' 1
## 
## Residual standard error: 48.08 on 16 degrees of freedom
## Multiple R-squared:  0.4372, Adjusted R-squared:  0.2613 
## F-statistic: 2.486 on 5 and 16 DF,  p-value: 0.07555
\end{verbatim}

\begin{verbatim}
## Response 1 :
## 
## Call:
## lm(formula = `1` ~ dist + plot, data = dat)
## 
## Residuals:
##      Min       1Q   Median       3Q      Max 
## -0.00973 -0.00752 -0.00576 -0.00015  0.56404 
## 
## Coefficients:
##              Estimate Std. Error t value Pr(>|t|)
## (Intercept) -0.001544   0.002501  -0.617    0.537
## dist        -0.001298   0.003857  -0.337    0.737
## plot         0.001307   0.003862   0.338    0.735
## 
## Residual standard error: 0.0284 on 1239 degrees of freedom
## Multiple R-squared:  0.0004468,  Adjusted R-squared:  -0.001167 
## F-statistic: 0.2769 on 2 and 1239 DF,  p-value: 0.7582
## 
## 
## Response 2 :
## 
## Call:
## lm(formula = `2` ~ dist + plot, data = dat)
## 
## Residuals:
##      Min       1Q   Median       3Q      Max 
## -0.21901 -0.00303  0.00765  0.01152  0.40945 
## 
## Coefficients:
##              Estimate Std. Error t value Pr(>|t|)  
## (Intercept) -0.001857   0.002496  -0.744   0.4571  
## dist        -0.007380   0.003849  -1.917   0.0554 .
## plot         0.007375   0.003854   1.913   0.0559 .
## ---
## Signif. codes:  0 '***' 0.001 '**' 0.01 '*' 0.05 '.' 0.1 ' ' 1
## 
## Residual standard error: 0.02834 on 1239 degrees of freedom
## Multiple R-squared:  0.004565,   Adjusted R-squared:  0.002958 
## F-statistic: 2.841 on 2 and 1239 DF,  p-value: 0.05874
\end{verbatim}

\begin{verbatim}
## Response 1 :
## 
## Call:
## lm(formula = `1` ~ dist + plot, data = datp)
## 
## Residuals:
##      Min       1Q   Median       3Q      Max 
## -0.22486 -0.13069 -0.03585  0.03691  0.54840 
## 
## Coefficients:
##             Estimate Std. Error t value Pr(>|t|)
## (Intercept) -0.08527    0.11517  -0.740    0.465
## dist        -0.07817    0.16403  -0.477    0.638
## plot         0.07863    0.16429   0.479    0.636
## 
## Residual standard error: 0.19 on 27 degrees of freedom
## Multiple R-squared:  0.02533,    Adjusted R-squared:  -0.04687 
## F-statistic: 0.3508 on 2 and 27 DF,  p-value: 0.7073
## 
## 
## Response 2 :
## 
## Call:
## lm(formula = `2` ~ dist + plot, data = datp)
## 
## Residuals:
##      Min       1Q   Median       3Q      Max 
## -0.29484 -0.14106 -0.01064  0.08475  0.39195 
## 
## Coefficients:
##             Estimate Std. Error t value Pr(>|t|)  
## (Intercept) -0.09838    0.10500  -0.937   0.3571  
## dist        -0.31294    0.14954  -2.093   0.0459 *
## plot         0.31280    0.14977   2.088   0.0463 *
## ---
## Signif. codes:  0 '***' 0.001 '**' 0.01 '*' 0.05 '.' 0.1 ' ' 1
## 
## Residual standard error: 0.1732 on 27 degrees of freedom
## Multiple R-squared:  0.1899, Adjusted R-squared:  0.1299 
## F-statistic: 3.165 on 2 and 27 DF,  p-value: 0.05823
\end{verbatim}

\begin{verbatim}
##              Df  Sum Sq Mean Sq F value   Pr(>F)    
## dist          1  354161  354161  42.242 4.05e-10 ***
## plot          1    1990    1990   0.237    0.627    
## Residuals   262 2196631    8384                     
## ---
## Signif. codes:  0 '***' 0.001 '**' 0.01 '*' 0.05 '.' 0.1 ' ' 1
\end{verbatim}

\begin{verbatim}
##              Df    Sum Sq Mean Sq F value Pr(>F)
## dist          1      1012    1012   0.006  0.938
## plot          1     32528   32528   0.196  0.658
## Residuals   974 161488210  165799
\end{verbatim}

\emph{Table S\#}. (A-D) Linear model and (E, F) ANOVA results for (A, D)
total mean, (B, E) \emph{B. quinata}, and (C, F) other taxon biomass
with respect to (A-C) stem density and (D-F) distance from primary
forest edge and plot ID.

\begin{verbatim}
## 
## Call:
## lm(formula = n ~ luz + incl + N + W, data = datp)
## 
## Residuals:
##     Min      1Q  Median      3Q     Max 
## -14.809  -8.642  -2.051   5.632  27.496 
## 
## Coefficients:
##               Estimate Std. Error t value Pr(>|t|)
## (Intercept) -3.341e+04  1.686e+05  -0.198    0.844
## luz          2.693e-02  5.339e-01   0.050    0.960
## incl         1.302e-01  1.377e-01   0.945    0.354
## N            3.941e+03  2.713e+03   1.453    0.159
## W            3.618e+00  1.965e+03   0.002    0.999
## 
## Residual standard error: 12.12 on 25 degrees of freedom
## Multiple R-squared:  0.1531, Adjusted R-squared:  0.01756 
## F-statistic:  1.13 on 4 and 25 DF,  p-value: 0.3652
\end{verbatim}

\begin{verbatim}
##             Df Sum Sq Mean Sq F value Pr(>F)  
## dist         1     54    54.3   0.387 0.5393  
## plot         1    483   483.5   3.440 0.0746 .
## Residuals   27   3795   140.6                 
## ---
## Signif. codes:  0 '***' 0.001 '**' 0.01 '*' 0.05 '.' 0.1 ' ' 1
\end{verbatim}

\begin{verbatim}
##                        Df Sum Sq Mean Sq F value Pr(>F)  
## poly(dist, degree = 2)  2    722   360.8   2.954 0.0698 .
## plot                    1    437   436.5   3.574 0.0699 .
## Residuals              26   3175   122.1                 
## ---
## Signif. codes:  0 '***' 0.001 '**' 0.01 '*' 0.05 '.' 0.1 ' ' 1
\end{verbatim}

\begin{verbatim}
##             Df Sum Sq Mean Sq F value Pr(>F)  
## dist         1    0.1    0.12   0.004 0.9489  
## plot         1  142.5  142.47   4.954 0.0383 *
## Residuals   19  546.4   28.76                 
## ---
## Signif. codes:  0 '***' 0.001 '**' 0.01 '*' 0.05 '.' 0.1 ' ' 1
\end{verbatim}

\begin{verbatim}
##             Df Sum Sq Mean Sq F value Pr(>F)  
## dist         1   22.6    22.6   0.329  0.571  
## plot         1  331.4   331.4   4.817  0.037 *
## Residuals   27 1857.4    68.8                 
## ---
## Signif. codes:  0 '***' 0.001 '**' 0.01 '*' 0.05 '.' 0.1 ' ' 1
\end{verbatim}

\emph{Table S\#}. Linear model and ANOVA results for stem density with
respect to distance from primary forest edge and plot ID.

\hypertarget{community-structure}{%
\subsubsection{\texorpdfstring{\emph{Community
structure}}{Community structure}}\label{community-structure}}

\begin{itemize}
\tightlist
\item
  diversity predicted by height and/or biomass (Pan et al 2018)
\end{itemize}

\begin{verbatim}
## 
## Call:
## lm(formula = r ~ h + kg14 + n, data = datp)
## 
## Residuals:
##    Min     1Q Median     3Q    Max 
## -3.363 -1.378 -0.082  1.567  4.169 
## 
## Coefficients:
##              Estimate Std. Error t value Pr(>|t|)    
## (Intercept) 17.546249   3.042766   5.767 4.51e-06 ***
## h           -0.749553   0.265174  -2.827  0.00893 ** 
## kg14         0.008226   0.007414   1.109  0.27741    
## n           -0.037015   0.033893  -1.092  0.28480    
## ---
## Signif. codes:  0 '***' 0.001 '**' 0.01 '*' 0.05 '.' 0.1 ' ' 1
## 
## Residual standard error: 2.139 on 26 degrees of freedom
## Multiple R-squared:  0.2952, Adjusted R-squared:  0.2139 
## F-statistic:  3.63 on 3 and 26 DF,  p-value: 0.026
\end{verbatim}

\emph{Table S\#}. Linear model results showing plot genus richness
explained by mean plot height.

\begin{itemize}
\tightlist
\item
  Other (non-richness) diversity measures, by distance to edge?
\end{itemize}

\hypertarget{results}{%
\section{Results}\label{results}}

``Large trees increase biomass by reducing richness {[}density{]} in a
regenerating tropical timber plantation''.

\hypertarget{census-structure-1}{%
\subsection{Census structure}\label{census-structure-1}}

Mean total plot stem density was explained by distance from primary
forest edge and plot. The relationship with distance from primary forest
edge was positively quadratic.

\includegraphics{osa_files/figure-latex/mapNdistPlot-1.pdf}

\begin{verbatim}
## Warning: Removed 1 rows containing missing values (geom_errorbar).
\end{verbatim}

\includegraphics{osa_files/figure-latex/mapNdistPlot-2.pdf}

\emph{Fig \#\#}. (A) Map showing total stem density and GPS coordinates
of census plots (21 x 21 m), and (B) plot showing stem density by
distance from primary forest edge.

Mean plot height was explained by mean plot specific wood density and
distance from primary forest edge. These were negative relationships,
which were further explained by a positive relationship between distance
from primary forest edge and mean plot specific wood density.

\hypertarget{canopy-structure-1}{%
\subsection{Canopy structure}\label{canopy-structure-1}}

\includegraphics{osa_files/figure-latex/HspgPlot-1.pdf}

\emph{Fig \#\#}. Plot showing relationship betwen mean plot (black,
individual trees in gray) tree heights and specific wood density.

\begin{verbatim}
## Warning: Removed 1 rows containing missing values (geom_errorbar).
\end{verbatim}

\includegraphics{osa_files/figure-latex/HspgDistPlot-1.pdf}

\begin{verbatim}
## Warning: Removed 1 rows containing missing values (geom_errorbar).
\end{verbatim}

\includegraphics{osa_files/figure-latex/HspgDistPlot-2.pdf}

\emph{Fig \#\#}. Plots showing the relationship between mean plot
(black, individual trees shown in gray) (A) heights and (B) specific
wood densities.

\hypertarget{biomass-structure}{%
\subsection{Biomass structure}\label{biomass-structure}}

Mean plot biomass was explained by distance from primary forest edge and
mean total plot stem density. The trend with edge distance was
negatively quadratic, while that with mean total plot stem density was
linearly negative. These relationships correspond with that between edge
distance and mean plot stem density, which was instead positively
quadratic (previous figure).

\emph{Taken together, these data show that mean total plot biomass was
lower closer to the primary forest, due to increases in stem density,
primarily with trees with low wood density and presumably faster growth
and reproduction rates (CITE?).}

Finally, there was an effect of East-West location on mean plot biomass,
but despite observing a downward slope on the southwest corner of the
parcel, this general East-West co-variation with slope did not
statistically explain mean plot biomass variation.

\includegraphics{osa_files/figure-latex/kgnPlot-1.pdf}

\emph{Fig \#\#}. Plot showing relationship between mean plot biomass and
mean plot stem density.

\begin{verbatim}
## Warning: Removed 1 rows containing missing values (geom_errorbar).
\end{verbatim}

\includegraphics{osa_files/figure-latex/kgDistPlot-1.pdf}

\emph{Fig \#\#}. Biomass (plot mean in black, individual trees in gray,
\emph{B. quinata} in orange) changes at various distance from primary
forest edge.

\hypertarget{community-structure-1}{%
\subsubsection{\texorpdfstring{\emph{Community
structure}}{Community structure}}\label{community-structure-1}}

\begin{itemize}
\tightlist
\item
  decide which plot represents anticipated stats\ldots{}(prob PCoA)
\end{itemize}

\includegraphics{osa_files/figure-latex/genDistPlot-1.pdf}

\emph{Fig \#\#}. Plot showing biomass partitioned by genus as a function
of distance from primary forest edge.

Plot richness was predicted by mean plot height, which supports previous
findings (Pan et al 2018).

\includegraphics{osa_files/figure-latex/ShDistPlot-1.pdf}

\begin{verbatim}
## Warning: Removed 1 rows containing missing values (geom_errorbar).
\end{verbatim}

\includegraphics{osa_files/figure-latex/ShDistPlot-2.pdf}

\emph{Fig \#\#}. Plot showing relationship between richness and (A) mean
plot tree height and (B) distance from primary forest edge.

\hypertarget{other-ongoing-analyses}{%
\subsection{\texorpdfstring{\emph{Other ongoing
analyses\ldots{}}}{Other ongoing analyses\ldots{}}}\label{other-ongoing-analyses}}

\begin{verbatim}
## `stat_bin()` using `bins = 30`. Pick better value with `binwidth`.
\end{verbatim}

\includegraphics{osa_files/figure-latex/kgDistHist-1.pdf}

\begin{verbatim}
## <ggproto object: Class FacetWrap, Facet, gg>
##     compute_layout: function
##     draw_back: function
##     draw_front: function
##     draw_labels: function
##     draw_panels: function
##     finish_data: function
##     init_scales: function
##     map_data: function
##     params: list
##     setup_data: function
##     setup_params: function
##     shrink: TRUE
##     train_scales: function
##     vars: function
##     super:  <ggproto object: Class FacetWrap, Facet, gg>
\end{verbatim}

\begin{verbatim}
## `stat_bin()` using `bins = 30`. Pick better value with `binwidth`.
\end{verbatim}

\includegraphics{osa_files/figure-latex/kgDistHist-2.pdf}

\begin{verbatim}
## `stat_bin()` using `bins = 30`. Pick better value with `binwidth`.
\end{verbatim}

\includegraphics{osa_files/figure-latex/kgDistHist-3.pdf}

\begin{verbatim}
## `stat_bin()` using `bins = 30`. Pick better value with `binwidth`.
\end{verbatim}

\includegraphics{osa_files/figure-latex/hDistHist-1.pdf}

\begin{verbatim}
## `stat_bin()` using `bins = 30`. Pick better value with `binwidth`.
\end{verbatim}

\includegraphics{osa_files/figure-latex/hDistHist-2.pdf}

\begin{verbatim}
## `stat_bin()` using `bins = 30`. Pick better value with `binwidth`.
\end{verbatim}

\includegraphics{osa_files/figure-latex/hDistHist-3.pdf}

\emph{Figs \#}. Histograms showing plot (A) biomass and (B) height
distributions by distance from primary forest edge.

\begin{itemize}
\tightlist
\item
  forest biomass is complex (Pan et al 2013)
\item
  but biomass distributions here are log-normal?!
\item
  check for \emph{power-law fits} of height and biomass histograms among
  edge plots
\end{itemize}

\hypertarget{codeplot-attic-dont-read}{%
\section{CODE/PLOT ATTIC, DONT READ}\label{codeplot-attic-dont-read}}

\#\#Fig1.Biomass,distance,genus

Fig 1a. Intermediate distances (150m from edge) tended (p=0.26) to show
higher biomass.

(Outliers seem to be relevant for variance at intermediate
distances\ldots{})

Lonchocarpus individual is huge. Remove?

\#\#FigA1.Vstem,distance

\#\#Bombacopsis biomass by distance

Figs 1b and c. (b) Biomass of plantation species significantly increased
with distance from undisturbed forest edge (p\textless{}0.05, R2=0.70),
with (b) no change in the number of \emph{Bombacopsis} stems
(p\textgreater{}0.05).

Together, Figs 1a and 1b show that biomass was higher in areas where the
plantation species \emph{Bombacopsis} stems grew smaller on average.

(This may suggest either competition or facilitation--check biomass
ditribution of major contributing genera\ldots{})

Fig 2a. Rank-biomass curves (by genus) significantly (but weakly) fit a
negative exponential (p=0.01, R2=0.05).

From Fig 1, highest biomass was in 150m, which this figure shows mostly
comes from 1 \emph{Lonchocarpus}, 1 \emph{Tabebuia}, 1 {[}/3{]}
\emph{Ficus}, 2 {[}/10{]} \emph{Hyeronima}, 1 {[}/2{]} \emph{Apeiba}, 4
{[}/23{]} \emph{Vochysia}, and 8 {[}/16{]} \emph{Bombacopsis} trees
(ranked sequentially). Excluding individual outlier tree sizes, it could
be

\#\#Vochysia biomass by distance

Fig 2b and c. {[}\ldots{}{]} Actually, it seems like loss of biomass may
not due to competition between \emph{Bombacopsis} and any single taxon,
given that the trends of lower richness with higher biomass observed
here seem to be due to singletons.

**Does this suggest a kind of portfolio effect, but at the level of
productive individuals (vs.~species), and in the opposite direction due
to competition?

This could be due to a lower chance of sampling species in an area with
a few big trees. -- What does the stem number distribution look like by
distance and/or genus?

Seems like it is true, that fewer individuals of rare species occurred
in plots that also had the most massive (and the least massive)
individual trees.

Fig \#a. Frequency distribution of biomass tends to appear log-normal.

DISCUSSION - explain hump b/c proximity on road side + -- lower richness
b/c earlier-successional (20y) - Vochysia is wind-dispersed; is avg
200m?

\hypertarget{community-composition-1}{%
\subsubsection{Community composition}\label{community-composition-1}}

\hypertarget{shannon-diversity}{%
\paragraph{Shannon diversity}\label{shannon-diversity}}

Fig 3. Richness-by-distance relationship significantly fits a negative
quadratic model (p\textless{}0.01, R2=0.22).

(Which {[}rare{]} species were lost?)

Together, Figs 1 and 3 show that higher biomass tended to occur in plots
with lower taxonomic richness.

Fig 4. Biomass-richness relationship marginally significantly fits a
straight line on a log-log plot (p=0.066, R2=0.63).

\#Conclusions Higher biomass tended to occur in areas that also showed
lower taxonomic richness at \textgreater{}10 years after abandonment of
the \emph{Bombacopsis} tree plantation.

\#\#Attic \#\#\#Vochysia


\end{document}
